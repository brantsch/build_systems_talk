\begin{frame}
	\frametitle{Was sind Build-Systeme?}
	\pause
	\Large
	Build-Systeme sind Werkzeuge, welche den Erstellungsprozess von Software automatisieren.
\end{frame}

\subsection{Motivation}
\begin{frame}
	\frametitle{Warum Build-Systeme?}
	\pause
	\textbf{$\rightarrow$ Erleichterung und Verbesserung der Arbeit}
	\begin{itemize}
		\pause
		\item Erstellung/Neuerstellung mit nur einem Kommando
		\pause
		\item projektübergreifend einheitliche Befehle (Konventionen für Build-System Konfiguration)
		\pause
		\item Unterstützung der Portierung auf verschiedene Plattformen
		\pause
		\item Beschleunigung des Erstellungsprozesses
			\begin{itemize}
			\pause
			\item Ausführung der benötigten Aktionen in rascher Folge ohne Pause (vgl. manuelles Eintippen von Kommandos)
			\pause
			\item Vermeidung unnötiger Erstellung/Neuerstellung
			\pause
			\item Möglichkeit der automatischen Parallelisierung der Erstellung (bis hin zur verteilten Erstellung auf Build-Farm)
			\end{itemize}
	\end{itemize}
\end{frame}
\subsection*{}

\begin{frame}
	\frametitle{Was tun Build-Systeme?}
	\begin{itemize}
		\pause
		\item Automatisierung des Übersetzungsprozesses
			\begin{itemize}
				\pause
				\item Verfolgung von Abhängigkeiten
				\pause
				\item Erzeugung benötigter Ressourcen
			\end{itemize}
		\pause
		\item Berücksichtigung von Konfiguration
			\begin{itemize}
			\pause
			\item z.B. Ein- und Ausschalten von Features
			\end{itemize}
		\pause
		\item Ausführung von Tests zur Sicherstellung des erfolgreichen Builds
		\begin{itemize}
			\pause
			\item Feature-Detection auf dem System (vor dem Build)
			\pause
			\item Prüfung des Build-Ergebnisses
		\end{itemize}
		\pause
		\item Paketierung und/oder Installation des Build-Ergebnisses
	\end{itemize}
\end{frame}

\begin{frame}
	\frametitle{Betrachtete Build-Systeme}
	\begin{itemize}
		\item GNU Make
		\item CMake
		\item OMake
		\item GNU Autotools
	\end{itemize}
\end{frame}
