\subsection{Was sind Build-Systeme?}
\begin{frame}
	\Large
	Build-Systeme sind Werkzeuge, welche den Erstellungsprozess von Software automatisieren.
\end{frame}

\begin{frame}
	\textbf{Was bedeutet das?}
	\begin{itemize}
		\pause
		\item Automatisierung des Übersetzungsprozesses
			\begin{itemize}
				\pause
				\item Verfolgung von Abhängigkeiten
				\pause
				\item Erzeugung benötigter Ressourcen
			\end{itemize}
		\pause
		\item Berücksichtigung von Konfiguration
			\begin{itemize}
			\item z.B. Ein- und Ausschalten von Features
			\end{itemize}
		\pause
		\item Ausführung von Tests zur Sicherstellung des erfolgreichen Builds
		\begin{itemize}
			\pause
			\item Feature-Detection auf dem System (vor dem Build)
			\pause
			\item Prüfung des Build-Ergebnisses
		\end{itemize}
		\pause
		\item Paketierung und/oder Installation des Build-Ergebnisses
	\end{itemize}
\end{frame}

\begin{frame}
	\textbf{Betrachtete Build-Systeme}
	\begin{tabular}{p{8em}p{24em}}
	GNU Make & Der Klassiker \\
	CMake & Ein Meta-Make für mehr Portabilität \\
	omake & \\
	\sout{Autohell} \linebreak GNU Autotools 
	\end{tabular}
\end{frame}
