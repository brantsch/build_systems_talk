\subsection{Bewertung von GNU Make}
\begin{frame}
	\frametitle{Stärken von GNU Make}
	\begin{itemize}
		\pause
		\item Einfachheit des Einstiegs
		\pause
		\item für viele Anwendungsfälle ausreichende Mächtigkeit
		\pause
		\item Flexibilität
	\end{itemize}
\end{frame}

\begin{frame}
	\frametitle{Schwächen von GNU Make}
	\begin{itemize}
		\pause
		\item Unbequemlichkeit des Debuggings/Fehlen eines Debuggers
		\pause
		\item Verlass auf das Änderungsdatum von Dateien zur Erkennung von Änderungen
		\pause
		\item Fehlen der Unterstützung für Out-Of-Place-Builds

			Projekte können (ohne eigene Hacks) nur im Projektverzeichnis gebaut werden
		\pause
		\item „Geschwätzigkeit“ von Makefiles für Standard-Aufgaben
		\pause
		\item Mangel an Portabilität
			\begin{itemize}
				\pause
				\item GNU Make ist \emph{nicht} das einzige Make
				\pause
				\item Eigenheiten der verschiedenen POSIX-Systeme und Make-Implementierungen
			\end{itemize}
	\end{itemize}
\end{frame}

\subsection{Dokumentation}
\begin{frame}
	\frametitle{Literatur zu GNU Make}
	\begin{itemize}
	\pause
	\item Das offizielle Handbuch zu GNU Make \nocite{GNU_Make_manual}
	\pause
	\item Managing Projects with GNU Make \nocite{OreillyMake}
	\pause
	\item Debugging Makefiles \nocite{DobbsDebuggingMakefiles}
	\end{itemize}
\end{frame}
