\subsection{Stärken von Make}
\begin{frame}
	\begin{itemize}
		\item Einfachheit des Einstiegs
		\item Flexibilität
	\end{itemize}
\end{frame}

\subsection{Schwächen von Make}
\begin{frame}
	\begin{itemize}
		\item Verlass auf das Änderungsdatum von Dateien zur Erkennung von Änderungen
		\item Fehlen der Unterstützung für Out-Of-Place-Builds

			Projekte können nur im Projektverzeichnis gebaut werden
		\item Boilerplate: häufig benötigte Funktionalität nicht eingebaut
			\begin{itemize}
				\item Konfiguration des zu erstellenden Projekts (\texttt{configure} müsste selbst geschrieben werden)
			\end{itemize}
		\item Mangel an Portabilität
			\begin{itemize}
				\item GNU Make ist \emph{nicht} das einzige Make
				\item Eigenheiten der POSIX-Systeme
			\end{itemize}
	\end{itemize}
\end{frame}

\subsection{Dokumentation}
\begin{frame}
	\begin{itemize}
	\item Das offizielle Handbuch zu GNU Make \cite{GNU_Make_manual}
	\item Managing Projects with GNU Make \cite{OreillyMake}
	\item Debugging Makefiles \cite{DobbsDebuggingMakefiles}
	\end{itemize}
\end{frame}
