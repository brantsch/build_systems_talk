\subsection{Bewertung von GNU Make}
\begin{frame}
	\frametitle{Stärken von GNU Make}
	\begin{itemize}
		\pause
		\item Einfachheit des Einstiegs
		\pause
		\item für viele Anwendungsfälle ausreichende Mächtigkeit
		\pause
		\item Flexibilität
		\pause
		\item Allgegenwärtigkeit auf Linux-Systemen
	\end{itemize}
\end{frame}

\begin{frame}
	\frametitle{Schwächen von GNU Make}
	\begin{itemize}
		\pause
		\item \sout{Unbequemlichkeit des Debuggings/Fehlen eines Debuggers}\footnote{\tiny\url{http://gmd.sourceforge.net/} J. Graham-Cummings interaktiver Makefile-Debugger}
		\pause
		\item Fehlende Unterstützung für simultanes Aktualisieren mehrerer Targets durch die selbe Regel
		\pause
		\item Verlass auf das Änderungsdatum von Dateien zur Erkennung von Änderungen
		\pause
		\item Fehlen der Unterstützung für Out-Of-Place-Builds

			Projekte können (ohne eigene Hacks) nur im Projektverzeichnis gebaut werden
		\pause
		\item „Geschwätzigkeit“ von Makefiles für Standard-Aufgaben
		\pause
		\item Mangel an Portabilität
			\begin{itemize}
				\pause
				\item GNU Make ist \emph{nicht} das einzige Make
				\pause
				\item Eigenheiten der verschiedenen POSIX-Systeme und Make-Implementierungen
			\end{itemize}
	\end{itemize}
\end{frame}

\subsection{Dokumentation}
\begin{frame}[allowframebreaks]
	\frametitle{Literatur zu GNU Make}
	\nocitegnumakebib{GNU_Make_manual}
	\nocitegnumakebib{OreillyMake}
	\nocitegnumakebib{DobbsDebuggingMakefiles}
	\nocitegnumakebib{JGC_GMD}
	\bibliographystylegnumakebib{plain}
	\bibliographygnumakebib{../build_systems}
\end{frame}
