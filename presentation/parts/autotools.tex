\begin{frame}
	\frametitle{Die GNU Autotools}
	\begin{itemize}
		\pause
		\item komplizierte Sammlung von Programmen
		\pause
		\item Ziel der Autotools: Unterstützung der Portierung von Software zwischen POSIX-Systemen
	\end{itemize}
	\begin{description}
		\pause
		\item[Autoconf]	unterstützt Erstellung portabler \texttt{configure} Skripte
		\pause
		\item[Automake]	unterstützt Erstellung portabler Makefiles mit standardisierten Targets
		\pause
		\item[Libtool] 	unterstützt portable Erstellung und Benutzung von Shared Libraries
	\end{description}
\end{frame}
%
%\begin{frame}
%	\frametitle{GNU Automake}
%	\begin{itemize}
%		\item Teil der GNU Autotools
%		\item ebenfalls ein \emph{Make-Make}: generiert Konfiguration für verschiedene Implementierungen von Make
%		\item \emph{cross-platform} so lange das System POSIX-kompatibel ist
%	\end{itemize}
%\end{frame}
