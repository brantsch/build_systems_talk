\begin{frame}
	\frametitle{Die GNU Autotools}
	\begin{description}
		\item[Autoconf]	unterstützt Erstellung portabler \texttt{configure} und \texttt{testsuite} Skripte
		\item[Automake]	unterstützt Erstellung portabler Makefiles mit standardisierten Targets
		\item[Libtool] 	unterstützt portable Erstellung und Benutzung von Shared Libraries
	\end{description}
\end{frame}
%
%\begin{frame}
%	\frametitle{GNU Automake}
%	\begin{itemize}
%		\item Teil der GNU Autotools
%		\item ebenfalls ein \emph{Make-Make}: generiert Konfiguration für verschiedene Implementierungen von Make
%		\item \emph{cross-platform} so lange das System POSIX-kompatibel ist
%	\end{itemize}
%\end{frame}

\begin{frame}
	\frametitle{Zusatzleistungen der Autotools}
	\textbf{Beispiele}
	\begin{description}[Abhängigkeitsverfolgung]
		\item[Portabilität] Unterstützung der Portierung zwischen POSIX kompatiblen Systemen
		\item[Abhängigkeitsverfolgung] automatisierte Ermittlung der Abhängigkeiten einer Datei (z.B. includes in C-Code)
		\item[mehr implizite Regeln] 
		\item[Ausführung von Tests] hauptsächlich Feature Detection auf Zielplattform
	%	\item[Standard-Targets] unter anderem für\begin{itemize}
	%			\item Erstellung von Programm und Dokumentation
	%			\item Installation/Deinstallation
	%			\item Aufräumen des Verzeichnisses
	%			\item Ausführen von Tests
	%		\end{itemize}
	\end{description}
\end{frame}

\subsection{Dokumentation}
\begin{frame}
	\frametitle{Literatur zu den GNU Autotools}
	\nociteautotoolsbib{CalcoteAutotools}
	\nociteautotoolsbib{AutomakeManual}
	\bibliographystyleautotoolsbib{plain}
	\bibliographyautotoolsbib{../build_systems.bib}
\end{frame}
