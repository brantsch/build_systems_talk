\subsection{Grundfunktionalität von GNU Make}
\begin{frame}
\cite{GNU_Make_manual} Auszug aus dem Handbuch zu GNU Make
\begin{quote}
The make utility automatically determines which pieces of a large program need to be recompiled, and issues commands to recompile them.
\end{quote}
\end{frame}

\subsection{Makefiles}
\begin{frame}
\frametitle{Makefile} Konfiguration von Make für ein bestimmtes Projekt

\textbf{Arten von Bestandteilen eines Makefiles}
\begin{itemize}
	\item Variable
	\item Macro: Verfahrensvorschrift, die in Recipes benutzt werden kann
	\item Rule: Entscheidungsvorschrift für die (Neu-)Erstellung ihres Targets abhängig von mit angegebenen Voraussetzungen
	\item Target: z.B. Datei, die Make erstellen/aktualisieren soll
	\item Recipe: Erstellungsvorschrift für ein Target
\end{itemize}
\end{frame}

\lstset{language=[gnu]make}
\begin{frame}[allowframebreaks,fragile]
\end{frame}

\subsection{Stärken von Make}
\begin{frame}
	\begin{itemize}
		\item Einfachheit des Einstiegs
		\item Flexibilität
	\end{itemize}
\end{frame}

\subsection{Schwächen von Make}
\begin{frame}
	\begin{itemize}
		\item Verlass auf das Änderungsdatum von Dateien zur Erkennung von Änderungen
		\item Fehlen der Unterstützung für Out-Of-Place-Builds

			Projekte können nur im Projektverzeichnis gebaut werden
		\item Boilerplate: häufig benötigte Funktionalität nicht eingebaut
			\begin{itemize}
				\item Konfiguration des zu erstellenden Projekts (\texttt{configure} müsste selbst geschrieben werden)
			\end{itemize}
		\item Mangel an Portabilität
			\begin{itemize}
				\item GNU Make ist \emph{nicht} das einzige Make
				\item Eigenheiten der POSIX-Systeme
			\end{itemize}
	\end{itemize}
\end{frame}

\subsection{Dokumentation}
\begin{frame}
	\begin{itemize}
	\item Das offizielle Handbuch zu GNU Make \cite{GNU_Make_manual}
	\item Managing Projects with GNU Make \cite{OreillyMake}
	\item Debugging Makefiles \cite{DobbsDebuggingMakefiles}
	\end{itemize}
\end{frame}
