\documentclass{beamer}

\usetheme{Montpellier}

\usepackage[ngerman]{babel}
\usepackage[utf8]{inputenc}
\usepackage[T1]{fontenc}
\usepackage{ulem}
\usepackage{fix-cm}
\usepackage[nounderscore]{syntax}
\usepackage{multibib}
\usepackage{minted}

%\usemintedstyle{tango}
%\usemintedstyle{murphy}
\usemintedstyle{friendly}
\newcommand{\mymintedstyle}{
	%\fontsize{4pt}{4pt}\selectfont
	\tiny
	\renewcommand{\theFancyVerbLine}{\sffamily
		\textcolor[gray]{0.3}{
		\oldstylenums{\arabic{FancyVerbLine}}}}
}
\newcommand{\inputmintedmake}[1]{
	\mymintedstyle
	\inputminted[linenos=true,stepnumber=5]{make}{#1}
}

\newcommand{\makeawesomefooter}{
	\setbeamertemplate{footline}{%
		\ttfamily%
		\fontsize{5pt}{5pt}\selectfont%
		\color{gray}%
		\hspace*{\fill}%
		[author] \csname beamer@shortauthor\endcsname%
		\hspace*{\fill}%
		[date] \csname beamer@shortdate\endcsname%
		\hspace*{\fill}%
		[git commit] \input{gitrev}%
		\hspace*{\fill}%
		\vspace{1pt}%
	}
}

%suppress the navigation bar
\beamertemplatenavigationsymbolsempty


%create parts of bibliography
\newcites{gnumakebib}{Literatur zu GNU Make}
\newcites{cmakebib}{Literatur zu CMake}
\newcites{autotoolsbib}{Literatur zu den GNU Autotools}

\begin{document}

	%% draft: disable \pause FIXME
	%\renewcommand{\pause}{}

	\section{Einführung}
	\begin{frame}
	\title{Build Systems für C, C++, etc.}
	\author{Peter Brantsch}
	\date{29.04.2014}
	\maketitle
\end{frame}

	\makeawesomefooter
	% vim: spell spelllang=de textwidth=80
\subsection{Motivation für den Einsatz eines Build-Systems} 

Als Werkzeuge der Software-Entwicklung sind Build-Systeme schon seit Jahrzehnten
nicht wegzudenken, denn sie erleichtern und verbessern die Arbeit der
Software-Entwickler erheblich, beziehungsweise erlauben den Entwicklern große
und komplexe Projekte überhaupt zu bewältigen. 
%
Durch den Einsatz eines Build-Systems sparen sich Entwickler viel Zeit und
Denk-Arbeit. Statt für die Erstellung bzw. Neuerstellung eines Projekts viele
Aufrufe diverser Programme manuell vornehmen zu müssen, sind beim Einsatz eines
Build-Systems nur noch wenige Aufrufe desselben nötig, in vielen Fällen auch
nur ein einziger.
%
Weitere Vereinfachungen, bestehend in projektübergreifend einheitlichen Befehlen
für das Build-System ergeben sich durch Konventionen für die Konfiguration, so
dass für das Erstellen vieler verschiedener Projekte einheitliche Aufrufe des
Build-Systems genügen.

Fortgeschrittene Build-Systeme unterstützen den Entwickler auch bei der
Portierung von Software-Projekten auf mehrere Plattformen. %FIXME

Außerdem beschleunigen Build-Systeme den Erstellungsprozess ungemein.
Build-Systeme führen die benötigten Aktionen in erheblich rascherer Folge aus
als ein Kommandos in eine Shell tippender Benutzer. 
%
Darüberhinaus sind selbst rudimentäre Build-Systeme bereits in der Lage unnötige
Erstellung/Neuerstellung von Teilen eines Programms zu vermeiden und sparen
dadurch Rechenzeit beim Erstellungsprozess.
%
Weiter beschleunigt werden kann die Erstellung durch automatische
Parallelisierung, bis hin zu verteilter Erstellung auf einer Build-Farm.

\subsection{Funktion eines Build-Systems}

Die offensichtlichste Funktion eines Build-Systems ist die Automatisierung der
Erstellung von Dateien. Dies schließt die Verfolgung von Abhängigkeiten ein,
d.h. dass das Build-System bei der Entscheidung ob eine Datei (neu) erstellt
werden soll berücksichtigt von welchen anderen Dateien diese abhängt.
%
Dabei gehen die Systeme rekursiv vor und erzeugen für die Erstellung einer Datei
benötigte Ressourcen, falls diese noch nicht vorliegen oder veraltet
\footnote{besitzen ihrerseits wiederum Abhängigkeiten, die neuer sind als sie
selbst} sind.

Ab einer gewissen Größe ergeben sich nennenswerte
Konfigurationsmöglichkeiten für Projekte, beispielsweise für das Ein- oder
Ausschalten von Features. Hinreichend fortgeschrittene Build-Systeme
unterstützen dies selbstverständlich. Konfigurationsvariablen werden
beispielsweise abhängig von Eigenschaften des Host-Systems gesetzt, oder auch
durch Eingaben des Benutzers, z.B. in einem Konfigurations-Dialog oder in einer
Konfigurationsdatei.

%
Zum Funktionsumfang fortgeschrittener Build-Systeme gehört unter anderem auch
die Ausführung von Tests zur Sicherstellung der erfolgreichen Erstellung. Wie
bereits erwähnt können Tests auch das Host-System untersuchen, z.B. um zu
überprüfen ob dieses über für einen erfolgreichen Build benötigte Features
verfügt. Nach abgeschlossener Erstellung kann des weiteren auch das
Build-Ergebnis getestet werden. Es bietet sich hier offensichtlich an, das
gerade erstellte Programm auf seine korrekte Funktion zu überprüfen.

Nicht notwendig, aber hilfreich, und in vielen Build-Systemen vorhanden ist
Unterstützung für Paketierung und/oder Installation des Build-Ergebnisses. So
sind einige Build-Systeme beispielsweise in der Lage, installierbare Pakete für
verschiedene Paketmanager/Installationsroutinen zu erstellen, oder
Build-Ergebnisse gleich auf dem Host-System in die richtigen Verzeichnisse zu
installieren.


	\section{GNU Make}
		\subsection{Grundfunktionalität von GNU Make}
\begin{frame}
	Auszug aus dem Handbuch zu GNU Make
	\begin{quote}
		The make utility automatically determines which pieces of a large
		program need to be recompiled, and issues commands to recompile them.
	\end{quote}
\end{frame}

		\subsection{Makefiles}
		\begin{frame}
\frametitle{Makefile} Konfiguration von Make für ein bestimmtes Projekt
\end{frame}

\begin{frame}
\frametitle{Bestandteile eines Makefiles}
\begin{description}[Variablendefinitionen] %length of options argument is measured to set width of items in description
	\item[Regeln] Beschreiben wann und wie Dateien (\emph{Targets}) abhängig von \emph{Prerequisites} erstellt werden sollen.
		\begin{description}
		\item[explizit] Targets benennen Dateien
		\item[implizit] Targets benennen Klassen von Dateien
		\end{description}
	\item[Direktiven] z.B. \begin{itemize}
			\item Einlesen anderer Makefiles
			\item Kontrollstrukturen für Entscheidungen
			\item Definition mehrzeiliger Variablen
		\end{itemize}
	\item[Variablendefinitionen] Zuweisungen von Zeichenketten zu Namen
	\item[Kommentare]
\end{description}
\end{frame}

\begin{frame}
	\frametitle{Kommentare}
	\begin{syntdiag}
		‘\#’
		\synt{text}
		\synt{newline}
	\end{syntdiag}
\end{frame}

\begin{frame}
	\frametitle{Strings}
	\begin{syntdiag}
		\begin{rep}
			\begin{stack}
				\tok{ASCII Zeichen ohne \$} \\
				\$\$ \\
				\synt{variablenreferenz} \\
				\synt{funktionsaufruf}
			\end{stack}
		\end{rep}
	\end{syntdiag}
\end{frame}

\begin{frame}
	\frametitle{Funktionsaufrufe}
	\begin{syntdiag}
		‘\$(’ \synt{name}
		\synt{whitespace}
		\begin{stack}
			\\
			\begin{rep}
				\synt{string} ‘,’
			\end{rep}
		\end{stack}
		\synt{string}
		‘)’
	\end{syntdiag}
\end{frame}

\begin{frame}
	\frametitle{Entscheidungs-Kontrollstrukturen}
	\begin{syntdiag}
		\synt{bedingung}
		\synt{string}
		\begin{stack}
			\\
			‘else’
			\synt{string}
		\end{stack}
		‘endif’
	\end{syntdiag}

	\textbf{Bedingungen}
	\begin{syntdiag}
		\begin{stack}
			‘ifeq’ \\
			‘ifneq’ \\
			‘ifdef’ \\
			‘ifndef’
		\end{stack}
		\begin{stack}
			‘(’ \synt{string} ‘,’ \synt{string} ‘)’ \\
			\begin{stack}
				‘''’ \synt{string} ‘''’ \\
				‘'’ \synt{string} ‘'‘
			\end{stack}
			\begin{stack}
				‘''’ \synt{string} ‘''‘ \\
				‘'’ \synt{string} ‘'’
			\end{stack}
		\end{stack}
	\end{syntdiag}
\end{frame}

\begin{frame}[allowframebreaks,fragile]
\frametitle{Variablen}

	\begin{description}[rekursiv expandierte Variablen]
		\item[rekursiv expandierte Variablen] keine Expansion in Definition, rekursive Expansion bei Referenz
		\item[einfach expandierte Variablen] Expansion erfolgt \emph{einmalig} bei Definition/Redefinition
		\item[automatische Variablen] Belegung durch Make
	\end{description}

	\framebreak

	\textbf{Variablenreferenz}
	\begin{syntdiag}
		\$ \begin{stack}
			\{ \synt{string} \} \\
			( \synt{string} )
		\end{stack}
	\end{syntdiag}

	\textbf{Definition/Redefinition einer Variable}
	\begin{syntdiag}
		\synt{variable name}
		\begin{stack}
			'=' \\
			':='
		\end{stack}
		\begin{rep}
			\synt{string}
		\end{rep}
	\end{syntdiag}
	\begin{syntdiag}
		‘define’
		\synt{variable name}
		‘=’
		\synt{newline}
		\begin{rep}
			\synt{string}
			\synt{newline}
		\end{rep}
		‘endef’
	\end{syntdiag}

	\textbf{Anhängen an eine Variable}
	\begin{syntdiag}
		\synt{variable name}
		‘+=’
		\begin{rep}
			\synt{string}
		\end{rep}
	\end{syntdiag}

	\vfill\eject
\end{frame}

\begin{frame}[allowframebreaks]
\frametitle{Regeln}
	\textbf{rule}
	\begin{syntdiag}
		\begin{rep}
			\synt{target}
		\end{rep}
		\begin{stack}
			‘:’ \\
			‘::’
		\end{stack}
		\begin{stack}
			\begin{rep}
				\synt{prerequisite}
			\end{rep} \\
		\end{stack}
		\begin{stack}
			\\
			\synt{newline}
			\begin{rep}
				\synt{recipe line}
			\end{rep}
		\end{stack}
	\end{syntdiag}

	\textbf{recipe line}
	\begin{syntdiag}
		\synt{tab}
		\begin{stack}
			\\
			\begin{rep}
				\synt{string}
				\synt{backslash}
				\synt{newline}
			\end{rep}
		\end{stack}
		\synt{string}
	\end{syntdiag}
\end{frame}

\begin{frame}
\frametitle{Targets und Prerequisites}
	\textbf{Target}
	\begin{syntdiag}
		\begin{stack}
			\synt{dateiname} \\
			\synt{pattern}
		\end{stack}
	\end{syntdiag}

	\textbf{Prerequisite}
	\begin{syntdiag}
		\begin{stack}
			\synt{anderes target} \\
			\synt{dateiname} \\
			\synt{pattern}
		\end{stack}
	\end{syntdiag}

	\textbf{Pattern}
	\begin{syntdiag}
		\begin{stack}
			\\
			\synt{string ohne whitespace}
		\end{stack}
		‘\%’
		\begin{stack}
			\\
			\synt{string ohne whitespace}
		\end{stack}
	\end{syntdiag}
\end{frame}

		%\begin{frame}[allowframebreaks]
%	\frametitle{Makefile für Worm000}
%	\inputmintedmake{../code/Worm000_Makefile}
%\end{frame}
%
%\begin{frame}[allowframebreaks]
%	\frametitle{Makefile für Worm030}
%	\inputmintedmake{../code/Worm030_Makefile}
%\end{frame}
%
%\begin{frame}[allowframebreaks]
%	\frametitle{Makefile für Worm050}
%	\inputmintedmake{../code/Worm050_Makefile}
%\end{frame}

\begin{frame}[allowframebreaks]
	\frametitle{Makefile für Worm080}
	\inputmintedmake{../code/Worm080_Makefile}
\end{frame}

\begin{frame}[allowframebreaks]
	\frametitle{Makefile.common}
	\inputmintedmake{../Makefile.common}
\end{frame}

\begin{frame}
	\frametitle{presentation/Makefile}
	\inputmintedmake{Makefile}
\end{frame}


		\subsection{Stärken von Make}
\begin{frame}
	\begin{itemize}
		\item Einfachheit des Einstiegs
		\item Flexibilität
	\end{itemize}
\end{frame}

\subsection{Schwächen von Make}
\begin{frame}
	\begin{itemize}
		\item Verlass auf das Änderungsdatum von Dateien zur Erkennung von Änderungen
		\item Fehlen der Unterstützung für Out-Of-Place-Builds

			Projekte können nur im Projektverzeichnis gebaut werden
		\item Boilerplate: häufig benötigte Funktionalität nicht eingebaut
			\begin{itemize}
				\item Konfiguration des zu erstellenden Projekts (\texttt{configure} müsste selbst geschrieben werden)
			\end{itemize}
		\item Mangel an Portabilität
			\begin{itemize}
				\item GNU Make ist \emph{nicht} das einzige Make
				\item Eigenheiten der POSIX-Systeme
			\end{itemize}
	\end{itemize}
\end{frame}

\subsection{Dokumentation}
\begin{frame}
	\begin{itemize}
	\item Das offizielle Handbuch zu GNU Make \cite{GNU_Make_manual}
	\item Managing Projects with GNU Make \cite{OreillyMake}
	\item Debugging Makefiles \cite{DobbsDebuggingMakefiles}
	\end{itemize}
\end{frame}


	\section{CMake}
		\nocitecmakebib{CMakeWiki}
\nocitecmakebib{CMakeDocumentation}
\nocitecmakebib{aosabook}
\bibliographystylecmakebib{plain}
\bibliographycmakebib{../build_systems.bib}

	\section{omake}
	\section{GNU Autotools}
		% vim: spell spelllang=de textwidth=80
Die Autotools \citeautotoolsbib{CalcoteAutotools} sind eine komplizierte
Sammlung von Programmen mit dem Ziel die Portierung von Software zwischen
UNIX-ähnlichen Systemen zu vereinfachen.
%
Zu diesem Zweck generieren die Autotools portable \texttt{Makefile}s und
\texttt{configure}-Skripte, welche nach Möglichkeit auf vielen UNIX-ähnlichen
Systemen funktionieren sollten. Außerdem bieten sie Entwicklern Funktionalität
für den portablen Umgang mit dynamischen Codebibliotheken.

\begin{description}
	%
	\item[Autoconf] Ein Generator für \texttt{configure}-Skripte.
	%
	\item[Automake] Ein Generator für portable parametrisierte
	\texttt{Makefile}s. \citeautotoolsbib{AutomakeManual}
	%
	\item[Libtool] Dieses Werkzeug dient zur portablen Handhabung von
	dynamischen Bibliotheken.
	%
\end{description}


	\section{Abschluss}
	\begin{frame}
	Danke für eure Aufmerksamkeit.
\end{frame}


	\section{Literatur}
	\bibliographystyle{plain}
	\bibliography{../build_systems}
\end{document}
