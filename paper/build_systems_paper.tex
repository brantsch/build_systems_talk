% vim: spell spelllang=de textwidth=80
\documentclass[a4paper,12pt]{article}
\usepackage[ngerman]{babel}
\usepackage[T1]{fontenc}
\usepackage[utf8]{inputenc}
\usepackage{fix-cm}
\usepackage{hyperref}
\usepackage{multibib}
\usepackage[table]{xcolor}
\usepackage{minted}
\usepackage[nounderscore]{syntax}
\usepackage{pifont}
%\usepackage{fancyhdr}

%draft watermark. Comment out in final release.
%\usepackage{eso-pic}
%\usepackage{graphicx}
%\makeatletter
%\AddToShipoutPicture{%
%	\raisebox{0.2\paperheight}[\paperheight][\paperwidth]{%
%		\rotatebox{45}{%
%			~~~~~~~~~~~~~
%			\textcolor[gray]{0.9}%
%			{\fontsize{6cm}{6cm}\selectfont{Entwurf}}%
%		}%
%	}%
%}
%\makeatother
%----------------

%create parts of bibliography
\newcites{gnumakebib}{Literatur zu GNU Make}
\newcites{cmakebib}{Literatur zu CMake}
\newcites{autotoolsbib}{Literatur zu den GNU Autotools}
\newcites{omakebib}{Literatur zu OMake}

%make hyperref look a little bit more beautiful
\hypersetup{%
	colorlinks=true,%
	linktoc=section,%
	citecolor=purple,%
	linkcolor=purple%
}

%re-style description item labels
\renewcommand*{\descriptionlabel}[1]{%
	\hspace\labelsep%
	\normalfont\underline{#1}%
}

%give us a nice checkmark
\newcommand{\cmark}{\ding{51}}
\newcommand{\xmark}{\ding{55}}
\newcommand{\tblcenter}[1]{
	\vspace*{\fill}%
	\hspace*{\fill}%
	#1%
	\hspace*{\fill}%
	\vspace*{\fill}%
}
\newcommand{\tblcmark}{\tblcenter{\cmark}}
\newcommand{\tblxmark}{\tblcenter{\xmark}}
%\newcommand{\tblcmark}{\cmark}
%\newcommand{\tblxmark}{\xmark}

%definitions
\author{Peter Brantsch}
\date{29. April 2014}

\begin{document}
	\pagenumbering{gobble}
	\begin{frame}
	\title{Build Systems für C, C++, etc.}
	\author{Peter Brantsch}
	\date{29.04.2014}
	\maketitle
\end{frame}

	\clearpage
	\tableofcontents

	\pagebreak
	\pagenumbering{arabic}
	\addtocounter{page}{2}

	%\fancyhead{}
	%\fancyfoot[C]{%
	%	\thepage
	%}
	%\renewcommand{\headrule}{}
	%\fancyhead[C]{%
	%	\tiny%
	%	\ttfamily%
	%	\color{gray}%
	%	[author]~\csname @author\endcsname%
	%	~[git commit]~\input{gitrev}%
	%	~[date]~\csname @date\endcsname%
	%}
	%\pagestyle{fancyplain}

	\section{Einführung}
		% vim: spell spelllang=de textwidth=80
\subsection{Motivation für den Einsatz eines Build-Systems} 

Als Werkzeuge der Software-Entwicklung sind Build-Systeme schon seit Jahrzehnten
nicht wegzudenken, denn sie erleichtern und verbessern die Arbeit der
Software-Entwickler erheblich, beziehungsweise erlauben den Entwicklern große
und komplexe Projekte überhaupt zu bewältigen. 
%
Durch den Einsatz eines Build-Systems sparen sich Entwickler viel Zeit und
Denk-Arbeit. Statt für die Erstellung bzw. Neuerstellung eines Projekts viele
Aufrufe diverser Programme manuell vornehmen zu müssen, sind beim Einsatz eines
Build-Systems nur noch wenige Aufrufe desselben nötig, in vielen Fällen auch
nur ein einziger.
%
Weitere Vereinfachungen, bestehend in projektübergreifend einheitlichen Befehlen
für das Build-System ergeben sich durch Konventionen für die Konfiguration, so
dass für das Erstellen vieler verschiedener Projekte einheitliche Aufrufe des
Build-Systems genügen.

Fortgeschrittene Build-Systeme unterstützen den Entwickler auch bei der
Portierung von Software-Projekten auf mehrere Plattformen. %FIXME

Außerdem beschleunigen Build-Systeme den Erstellungsprozess ungemein.
Build-Systeme führen die benötigten Aktionen in erheblich rascherer Folge aus
als ein Kommandos in eine Shell tippender Benutzer. 
%
Darüberhinaus sind selbst rudimentäre Build-Systeme bereits in der Lage unnötige
Erstellung/Neuerstellung von Teilen eines Programms zu vermeiden und sparen
dadurch Rechenzeit beim Erstellungsprozess.
%
Weiter beschleunigt werden kann die Erstellung durch automatische
Parallelisierung, bis hin zu verteilter Erstellung auf einer Build-Farm.

\subsection{Funktion eines Build-Systems}

Die offensichtlichste Funktion eines Build-Systems ist die Automatisierung der
Erstellung von Dateien. Dies schließt die Verfolgung von Abhängigkeiten ein,
d.h. dass das Build-System bei der Entscheidung ob eine Datei (neu) erstellt
werden soll berücksichtigt von welchen anderen Dateien diese abhängt.
%
Dabei gehen die Systeme rekursiv vor und erzeugen für die Erstellung einer Datei
benötigte Ressourcen, falls diese noch nicht vorliegen oder veraltet
\footnote{besitzen ihrerseits wiederum Abhängigkeiten, die neuer sind als sie
selbst} sind.

Ab einer gewissen Größe ergeben sich nennenswerte
Konfigurationsmöglichkeiten für Projekte, beispielsweise für das Ein- oder
Ausschalten von Features. Hinreichend fortgeschrittene Build-Systeme
unterstützen dies selbstverständlich. Konfigurationsvariablen werden
beispielsweise abhängig von Eigenschaften des Host-Systems gesetzt, oder auch
durch Eingaben des Benutzers, z.B. in einem Konfigurations-Dialog oder in einer
Konfigurationsdatei.

%
Zum Funktionsumfang fortgeschrittener Build-Systeme gehört unter anderem auch
die Ausführung von Tests zur Sicherstellung der erfolgreichen Erstellung. Wie
bereits erwähnt können Tests auch das Host-System untersuchen, z.B. um zu
überprüfen ob dieses über für einen erfolgreichen Build benötigte Features
verfügt. Nach abgeschlossener Erstellung kann des weiteren auch das
Build-Ergebnis getestet werden. Es bietet sich hier offensichtlich an, das
gerade erstellte Programm auf seine korrekte Funktion zu überprüfen.

Nicht notwendig, aber hilfreich, und in vielen Build-Systemen vorhanden ist
Unterstützung für Paketierung und/oder Installation des Build-Ergebnisses. So
sind einige Build-Systeme beispielsweise in der Lage, installierbare Pakete für
verschiedene Paketmanager/Installationsroutinen zu erstellen, oder
Build-Ergebnisse gleich auf dem Host-System in die richtigen Verzeichnisse zu
installieren.


	\section{GNU Make}
		% vim: spell spelllang=de textwidth=80
%\section{Einführung}
Das Programm \texttt{make} entstand im Umfeld von UNIX und ist Teil des POSIX
Standards.  Seine Aufgabe ist zu erkennen welche Teile eines großen Programms
sich geändert haben, und die Kommandos abzusetzen um diese zu aktualisieren.
\citegnumakebib{GNU_Make_Manual}
%
Im Folgenden soll von GNU Make die Rede sein, welches aufgrund der
Popularität von GNU/Linux heutzutage wohl das verbreitetste \texttt{make}
sein dürfte.

		% vim: spell spelllang=de textwidth=80
\subsection{Makefiles}

Beim Aufruf sucht \texttt{make} im aktuellen Arbeitsverzeichnis nach seiner
Konfiguration in Form eines \texttt{Makefile}.

\subsubsection{Bestandteile}

\begin{description}
%
	\item[Regeln] Beschreiben wann und wie Dateien (\emph{Targets} der Regel)
	abhängig von \emph{Prerequisites} erstellt werden sollen.
	%
	\begin{description}
		\item[explizit] Targets benennen Dateien
		\item[implizit] Targets benennen Klassen von Dateien
	\end{description}
	%
%
	\item[Direktiven] unter anderem:
	%
	\begin{itemize}
		\item Einlesen anderer Makefiles
		\item Kontrollstrukturen für Entscheidungen
		\item Definition mehrzeiliger Variablen
	\end{itemize}
%
	\item[Variablendefinitionen] Zuweisungen von Zeichenketten zu Namen
%
	\item[Kommentare] von \texttt{make} zu ignorierender Text
\end{description}

\subsubsection{Syntax}

	\subsubsection*{Kommentare}  
		%
		\begin{syntdiag}
	‘\#’
	\synt{text}
	\synt{newline}
\end{syntdiag}

		%
		Sobald \texttt{make} ein `\#'-Zeichen liest, ignoriert es den Rest der Zeile
		bis einschließlich des Zeilenumbruchs.

	\subsubsection*{Strings}  
		%
		\begin{syntdiag}
	\begin{rep}
		\begin{stack}
			\tok{ASCII Zeichen ohne \$} \\
			\$\$ \\
			\synt{variablenreferenz} \\
			\synt{funktionsaufruf}
		\end{stack}
	\end{rep}
\end{syntdiag}

		%
		GNU Make kennt keine anderen Datentypen als Strings.

	\subsubsection*{Funktionsaufrufe}  \begin{syntdiag}
	‘\$(’ \synt{name}
	\synt{whitespace}
	\begin{stack}
		\\
		\begin{rep}
			\synt{string} ‘,’
		\end{rep}
	\end{stack}
	\synt{string}
	‘)’
\end{syntdiag}

		%
		GNU Make stellt eine Sammlung von Hilfsfunktionen zur Verfügung, welche
		diversen Zwecken dienen, wie beispielsweise der Manipulation von
		Strings, der Umsetzung weiterer Bedingungs-Konstrukte, wiederholten
		Auswertung von Ausdrücken, etc.  \footnote{Siehe auch: Kapitel 8 (Functions
		for Transforming Text) in der offiziellen Dokumentation von GNU Make
		\citegnumakebib{GNU_Make_Manual}}

		Sowohl Funktionsaufrufe als auch Variablenreferenzen sind gleichermaßen
		mit runden Klammern wie mit geschweiften Klammern möglich, allerdings
		empfiehlt der Autor sich für jeweils eine Art von Klammern zu
		entscheiden und dabei zu bleiben.

	\pagebreak
	\subsubsection*{Entscheidungs-Kontrollstrukturen}	
		%
		\begin{syntdiag}
	\synt{bedingung}
	\synt{string}
	\begin{stack}
		\\
		\begin{stack}
			‘else’ \synt{string} \\
			\begin{rep}
				\synt{bedingung}
				\synt{string}
			\end{rep}
		\end{stack}
	\end{stack}
	‘endif’
\end{syntdiag}


	\subsubsection*{Bedingungen}  
		%
		\begin{syntdiag}
	\begin{stack}
		‘ifeq’ \\
		‘ifneq’ 
	\end{stack}
	\begin{stack}
		‘(’ \synt{string} ‘,’ \synt{string} ‘)’ \\
		\begin{stack}
			‘''’ \synt{string} ‘''’ \\
			‘'’ \synt{string} ‘'‘
		\end{stack}
		\begin{stack}
			‘''’ \synt{string} ‘''‘ \\
			‘'’ \synt{string} ‘'’
		\end{stack}
	\end{stack}
\end{syntdiag}

\begin{syntdiag}
	\begin{stack}
		‘ifdef’ \\
		‘ifndef’
	\end{stack}
	\synt{variable~name}
\end{syntdiag}


	\subsubsection*{Variablenreferenzen}  
		%
		\begin{syntdiag}
	\$ \begin{stack}
		\{ \synt{variable~name} \} \\
		( \synt{variable~name} )
	\end{stack}
\end{syntdiag}

		%
		\fcolorbox{red}{white}{
			\parbox{\linewidth-4\fboxsep}{
				Achtung: Referenzen auf undefinierte Variablen
				werden \emph{ohne Fehlermeldung} zu leeren Strings aufgelöst!
			}
		}
		\noindent
		Dieses Verhaltens sollte man sich bei der Fehlersuche in einem
		komplizierteren \texttt{Makefile} bewusst sein.

	\subsubsection*{Definition/Redefinition einer Variable}
		%
		\begin{syntdiag}
	\synt{variable~name}
	\begin{stack}
		‘=’ \\
		‘:=’
	\end{stack}
	\begin{rep}
		\synt{string}
	\end{rep}
	\synt{newline}
\end{syntdiag}
\begin{syntdiag}
	‘define’
	\synt{variable~name}
	\begin{stack}
		‘=’\\
		‘:=’
	\end{stack}
	\synt{newline}
	\begin{rep}
		\synt{string}
		\synt{newline}
	\end{rep}
	‘endef’
	\synt{newline}
\end{syntdiag}

		%
		\texttt{make} unterscheidet zwei Arten von Variablen, denen Werte
		explizit zugewiesen werden können: \emph{rekursiv expandierte Variablen}
		und \emph{einfach expandierte Variablen}.
		%
		Bei der Definition einer rekursiv expandierten Variable geschieht noch
		keinerlei Expansion von Variablenreferenzen. Diese erfolgt erst, wenn
		die Variable referenziert wird. Dann werden, wie der Name schon
		andeutet, Referenzen rekursiv expandiert.
		%
		Referenziert man hingegen in der Definition einer einfach expandierten
		Variable andere Variablen, so erfolgt die Expansion dieser Referenzen
		sofort. Gelegentlich ist das hilfreich, beispielsweise wenn man in der
		Definition einer Variable diese selbst referenzieren
		will\footnote{verwendete man eine rekursiv expandierte Variable
		entstünde hier eine Endlosschleife}, meistens werden aber normale,
		rekursiv expandierte Variablen genügen.
		
		Des Weiteren existieren auch sogenannte \emph{automatische Variablen},
		welche nicht explizit zugewiesen werden können, sondern ihre Werte von
		\texttt{make} erhalten.

	\subsubsection*{Anhängen an eine Variable}
		%
		\begin{syntdiag}
	\synt{variable~name}
	‘+=’
	\begin{rep}
		\synt{string}
	\end{rep}
\end{syntdiag}

		%	
		Der aufmerksame Leser wird beim Lesen des Beispiel-\texttt{Makefile}s
		auf Seite~\pageref{subsubsection:examplemakefile} feststellen, dass
		auch das Anhängen an undefinierte Variablen möglich ist. Es hat den
		selben Effekt wie eine normale Zuweisung.

	\subsubsection*{Regeln}
		%
		\begin{syntdiag}
	\begin{rep}
		\synt{target}
	\end{rep}
	\begin{stack}
		‘:’ \\
		‘::’
	\end{stack}
	\begin{stack}
		\begin{rep}
			\synt{prerequisite}
		\end{rep} \\
	\end{stack}
	\begin{stack}
		\\
		\synt{newline}
		\begin{rep}
			\synt{recipe line}
		\end{rep}
	\end{stack}
\end{syntdiag}

		%
		\textit{Recipe Lines} \begin{syntdiag}
	\begin{stack}
		\\
		\begin{rep}
			\synt{tab}
			\synt{string}
			‘\textbackslash’
			\synt{newline}
		\end{rep}
	\end{stack}
	\synt{tab}
	\synt{string}
	\synt{newline}
\end{syntdiag}

		%
		\textit{Targets} \begin{syntdiag}
	\begin{stack}
		\synt{dateiname} \\
		\synt{pattern}
	\end{stack}
\end{syntdiag}

		%
		\textit{Prerequisites} \begin{syntdiag}
	\begin{stack}
		\synt{anderes~target} \\
		\synt{dateiname} \\
		\synt{pattern}
	\end{stack}
\end{syntdiag}

		%
		\clearpage
		%
		\textit{Patterns} \begin{syntdiag}
	\begin{stack}
		\\
		\synt{string~ohne~whitespace}
	\end{stack}
	‘\%’
	\begin{stack}
		\\
		\synt{string~ohne~whitespace}
	\end{stack}
\end{syntdiag}


		Regeln bestimmen wie Dateien, die sogenannten \emph{Targets} einer
		Regel, zu aktualisieren sind, und von welchen anderen Dateien sie
		abhängen.
		
		Ist eine der Abhängigkeiten der Datei (im Syntaxdiagramm: Prerequisites)
		neuer als die Datei, so betrachtet \texttt{make} die Datei als veraltet
		und wird das Rezept ausführen um die Datei neu zu erstellen. Hierzu wird
		es jede Zeile (oben \textit{recipe line} genannt) von der Shell
		ausführen lassen.

		Recipe Lines sind in diesem Kontext als logische Zeilen zu verstehen.
		Eine logische Zeile kann sich über mehrere tatsächliche Zeilen
		erstrecken, falls alle Zeilenumbrüche, mit Ausnahme dessen der die
		logische Zeile beendet, mit Backslashes escaped werden.


		\clearpage
		\subsubsection{Ein Beispiel-\texttt{Makefile}}
			\label{subsubsection:examplemakefile}

			Hierbei handelt es sich um das \texttt{Makefile}, welches uns
			bereits im GdP1-Praktikum im ersten Semester begegnet ist.
			\footnote{Erstellt von Prof. Dr. Franz Regensburger} Damals diente
			es zur Übersetzung der Testate ab einschließlich \texttt{Worm080}.
			Mit der vorliegenden Beschreibung der Syntax und Funktionsweise von
			\texttt{make} sollte es leicht zu verstehen sein.

			{
			\footnotesize
			\inputminted[linenos=true,stepnumber=5]{make}{../code/Worm080_Makefile}
			}
		\clearpage

\subsection{Bedienung}

	Ruft man \texttt{make} ohne Kommandozeilenargumente in einem Ordner auf, in
	dem ein \texttt{Makefile} existiert, so wird es dieses einlesen und das
	erste Target erstellen.
	%
	Um explizit die Erstellung eines/mehrerer Targets anzufordern genügt es
	deren Namen als Kommandozeilenargumente anzugeben.

	Die hilfreichen Kommandozeilenargumente sind aber nicht auf die Auswahl von
	Targets beschränkt. Durch Hinzufügen der Option \texttt{--dry-run} bzw.
	\texttt{-n} zu einem Aufruf von \texttt{make} lassen sich die normalerweise
	zur Erstellung eines Targets ausgeführten Befehle schnell und einfach
	prüfen, da die genannte Option bewirkt dass diese nicht ausgeführt, sondern
	nur ausgegeben werden.
	%
	Eine weitergehende Beschreibung der Ausführung von \texttt{make} findet sich
	z.B. in der zugehörigen \texttt{man}-Page oder der offiziellen Dokumentation
	\citegnumakebib{GNU_Make_Manual}.

		% vim: spell spelllang=de textwidth=80
\section{Abschluss}

Nichts auf dieser Welt hat nur Vor- oder Nachteile. Im Folgenden soll nun kurz
auf die Vor- und Nachteile von GNU Make eingegangen werden.

\subsection{Stärken}

% Allgegenwärtigkeit auf POSIX-Systemen
Für \texttt{make} spricht zunächst ein mal dessen weite Verbreitung. Möchte man
sich der Verfügbarkeit eines Build-Systems für ein beliebiges UNIX(-ähnliches)
System sicher sein, so kann man getrost zu \texttt{make} greifen, denn dieses
ist Teil des POSIX-Standards\citegnumakebib{OpenGroupBaseSpec}, was jedoch -- in einem kleinen Vorgriff auf den
Abschnitt über Nachteile von \texttt{make} -- nicht den Eindruck vermitteln
sollte \texttt{Makefiles} wären immer portabel.

% Einfachheit des Einstiegs für viele Anwendungsfälle ausreichende Mächtigkeit
Die verglichen mit einigen anderen Build-Systemen rudimentäre Natur von
\texttt{make} macht des Weiteren den Einstieg leicht. Das \texttt{Makefile} für
ein kleines Projekt ist schnell geschrieben oder noch schneller kopiert.

% Flexibilität
Für viele Standard-Aufgaben ist \texttt{make} außerdem flexibel genug. Die
integrierten Funktionen sowie die Möglichkeit beliebige Programme aufzurufen
erlauben den Einsatz von \texttt{make} für diverseste Aufgaben bei denen Dateien
abhängig von anderen Dateien aktualisiert werden sollen.

\subsection{Schwächen} 

Das bisher besprochene Build-System ist bei weitem nicht der Weisheit letzter
Schluss. Deswegen wird sich auch \autoref{chapter:advancedbs} mit
Werkzeugen befassen, die deutlich fortgeschrittener sind als \texttt{make}.
Nicht ohne Grund ist auch die Liste der Schwächen deutlich länger als die der
Stärken.

%Debugging
Läuft ein Erstellungsprozess mit \texttt{make} nicht wie erwartet ab, ist man in
der Verlegenheit ein \texttt{Makefile} debuggen zu müssen. Wie dies geschehen
kann ist zwar recht gut dokumentiert \citegnumakebib{DobbsDebuggingMakefiles},
und es existiert sogar eine Art interaktiver Debugger \citegnumakebib{JGC_GMD},
dennoch ist dies bei weitem nicht so bequem wie das Debugging richtiger
Programmiersprachen.

Gelegentlich könnte man in die Verlegenheit kommen, in einem \texttt{Makefile}
Programme aufzurufen, welche mehrere Dateien gleichzeitig
verändern.\footnote{\LaTeX~z.B. erzeugt einen Wust von Dateien, von
denen einige von externen Tools gelesen werden müssen um ein Dokument korrekt zu
erzeugen.}%
Zu Ärger und/oder Enttäuschung führt in solchen Fällen \texttt{make}s fehlende
Unterstützung für Regeln die mehrere Targets gleichzeitig aktualisieren.

%Änderungserkennung
Manchmal wird sich auch die Änderungserkennung, welche sich auf das
Änderungsdatum der Dateien verlässt, als unpraktisch erweisen, denn oft genug
bedeutet ein neuer Wert des Änderungsdatums nicht auch eine Änderung des
Dateiinhalts. 
%Gravierender wirkt sich diese Schwäche im Zusammenhang mit
%Netzwerkdateisystemen und einer falsch gehenden Systemuhr aus.

%Geschwätzigkeit
Aus der rudimentären Natur von \texttt{make}, die seine Grundzüge so einfach und
leicht erlernbar macht, erwächst auch ein Nachteil, nämlich eine gewisse
„Geschwätzigkeit“ von \texttt{Makefile}s für Standardaufgaben. Da gewisse
Targets (z.B. \texttt{install}, \texttt{clean})
\citegnumakebib{GNUCodingStandards} oft benötigt werden, aber nicht eingebaut
sind, müssen sie jedes mal inklusive Regel und Rezept vom Entwickler geschrieben
werden.

%keine Out-Of-Source Builds
Weiterhin sind keine \emph{Out-Of-Source Builds} wie z.B. die gleichzeitige
Erstellung eines normalen und eines Debug-Builds in separaten Verzeichnissen
außerhalb des Quellcodeverzeichnisses möglich, d.h. die Erstellung
von Programmen muss, falls man nicht gewillt ist unschöne Workarounds zu
programmieren, im Quellcode-Verzeichnisbaum erfolgen.

%Mangel an Portabilität
Darüberhinaus besteht ein Mangel an Portabilität. Zwar ist \texttt{make}
standardisiert, aber Implementierungen wie z.B. GNU Make definieren umfangreiche
Erweiterungen.  
%
Selbst wenn es gelingt ein \texttt{Makefile} zu schreiben, das
in allen Implementierungen von \texttt{make} gültig ist, welche auf den durch
ein Projekt unterstützten Plattformen laufen, können diese Systeme immer noch
genügend Eigenheiten haben, wie beispielsweise verschiedene Arten mit
dynamischen Bibliotheken umzugehen.


	\section{Fortgeschrittene Build-Systeme}
		\label{chapter:advancedbs}
		%% vim: spell spelllang=de textwidth=80
Die im Folgenden vorgestellten Build-Systeme bedeuten einen Fortschritt
gegenüber von \texttt{make}, da sie Probleme lösen, welche \texttt{make} nicht
oder nur unzureichend löst.

		\subsection{OMake}
			\nociteomakebib{OMakeManual}
\bibliographystyleomakebib{plain}
\bibliographyomakebib{../build_systems.bib}

		\subsection{GNU Autotools}
			% vim: spell spelllang=de textwidth=80
Die Autotools \citeautotoolsbib{CalcoteAutotools} sind eine komplizierte
Sammlung von Programmen mit dem Ziel die Portierung von Software zwischen
UNIX-ähnlichen Systemen zu vereinfachen.
%
Zu diesem Zweck generieren die Autotools portable \texttt{Makefile}s und
\texttt{configure}-Skripte, welche nach Möglichkeit auf vielen UNIX-ähnlichen
Systemen funktionieren sollten. Außerdem bieten sie Entwicklern Funktionalität
für den portablen Umgang mit dynamischen Codebibliotheken.

\begin{description}
	%
	\item[Autoconf] Ein Generator für \texttt{configure}-Skripte.
	%
	\item[Automake] Ein Generator für portable parametrisierte
	\texttt{Makefile}s. \citeautotoolsbib{AutomakeManual}
	%
	\item[Libtool] Dieses Werkzeug dient zur portablen Handhabung von
	dynamischen Bibliotheken.
	%
\end{description}

		\subsection{CMake}
			\nocitecmakebib{CMakeWiki}
\nocitecmakebib{CMakeDocumentation}
\nocitecmakebib{aosabook}
\bibliographystylecmakebib{plain}
\bibliographycmakebib{../build_systems.bib}

		\clearpage
		\subsection{Vergleich}
			% vim: spell spelllang=de
\newenvironment{bstable}{%
	\definecolor{lightgray}{gray}{0.9}%
	\rowcolors{1}{white}{lightgray}%
	\begin{tabular}{%
		p{0.2\textwidth}%
		|%
		p{0.16\textwidth}%
		p{0.16\textwidth}%
		p{0.16\textwidth}%
		p{0.16\textwidth}%
	}%
	& \textbf{(GNU) Make} & \textbf{OMake} & \textbf{CMake} & \textbf{Autotools} \\ \hline%
}{%
	\end{tabular}%
}

Die folgende Übersicht ist bei weitem nicht erschöpfend. Die für den Vergleich
herangezogenen Features wurden vom Autor für wichtig und/oder praktisch
erachtet.

\noindent
\begin{bstable}
	Plattfor\-men 				 & POSIX & von OCaml unterstützte & POSIX, Windows, OS/2, etc. & POSIX \\
Unterstützung bei Cross-Compilation & \tblxmark & \tblxmark & \tblcmark & \tblcmark \\
Unterstützung bei Portierung & \tblxmark & \tblcmark & \tblcmark & \tblcmark \\

	interaktive Konfigu\-ration des Projekts 	& \tblxmark & \tblxmark & \tblcmark & \tblcmark \\
Autokon\-figuration 						& \tblxmark & \tblcmark & \tblcmark & \tblcmark \\
Test\-suite Unterstützung 					& \tblxmark & \tblxmark & \tblcmark & \tblcmark \\

	Erkennung von Änderungen 	& Zeit\-stempel & MD5 Digest & \tblcenter{???} & Zeitstempel\\
Abhängigkeits\-analyse		& \tblxmark & \tblcmark & \tblcmark & \tblcmark \\

	Parallele Erstellungs\-schritte & \tblcmark	& \tblcmark	& \tblcmark & \tblcmark \\
Fort\-schritts\-anzeige 	& \tblxmark & \tblcmark & \tblcmark & \tblxmark \\
Out-Of-Source Builds 		& \tblxmark & \tblcmark & \tblcmark & \tblcmark \\

\end{bstable}

\begin{itemize}
%
\item Unter \emph{Cross-Compilation} versteht man die Übersetzung von Software
auf einem System dessen Architektur von der des Systems auf welchem die Software
ausgeführt werden soll verschieden ist.
%
\item Im Zusammenhang dieses Vergleichs steht \emph{Unterstützung bei der
Portierung} für die Abstraktion von Eigenheiten der Zielsysteme durch das
Build-System, z.B. des Umgangs mit Bibliotheken.
%
\item Bei der \emph{interaktiven Konfiguration des Projekts} erfragt das
Build-System die gewünschten Werte für Konfigurationsvariablen über ein
Dialog-Interface von seinem Anwender.
%
\item \emph{Autokonfiguration} ist die automatische Erkennung von und Anpassung
an Gegebenheiten auf dem Zielsystem.
%
\item Build-Systeme, welche \emph{Testsuites} unterstützen, erlauben die
Ausführung von Tests des Build-Ergebnisses.
%
% FIXME: Änderungserkennung erklären
%
\item Eine Alternative zum mühsamen und fehleranfälligen Eintragen von
Abhängigkeiten per Hand ist die \emph{Abhängigkeitsanalyse}, bei welcher das
Build-System auswertet von welchen anderen Dateien eine Datei abhängt, z.B.
welche Header-Dateien eine C-Quellcodedatei inkludiert.
%
\item Wichtig besonders für größere Projekte sind \emph{parallele
Erstellungsschritte}. Ist dem Build-System per Kommandozeilenargument oder
Konfigurationsoption erlaubt, die Übersetzung zu parallelisieren, so wird es
versuchen voneinander unabhängige Erstellungsschritte parallel auszuführen.
Insbesondere auf Mehrprozessorsystemen, welche heutzutage sehr verbreitet sind,
lässt sich hierdurch die Erstellung z.T. deutlich beschleunigen. Die logische
Fortsetzung ist die verteilte Erstellung auf einer Build-Farm.
%
\item \emph{Out-Of-Source Builds}\footnote{gelegentlich auch \emph{Out-Of-Place
Builds} genannt} sind Erstellungen, welche außerhalb des Quellcodeverzeichnisses
eines Projekts erfolgen. Bei diesen liegen die Build-Konfiguration und das
Build-Ergebnis in einem vom Quellcodeverzeichnis separaten Verzeichnis. Sie sind
beispielsweise hilfreich wenn man gleichzeitig einen Build mit Debug-Symbolen
haben möchte und einen ohne.
%
\end{itemize}

\subsection*{Fazit}
%
Da CMake deutlich weniger kompliziert zu bedienen ist als die Autotools,
allerdings einen ähnlichen Funktionsumfang bietet, sollte man nach Ansicht des
Autors Abstand von den Autotools halten. Neben diesen wirkt CMake deutlich
fortschrittlicher, und in \textit{The Architecture of Open Source Applications}
\citecmakebib{aosabook} wird auch CMake als Ersatz für den alternden Ansatz mit
autoconf/libtool bezeichnet.

OMake sollte mit Vorsicht genossen werden, denn zumindest dem Autor fiel es
schwer Literatur (außer der offiziellen Dokumentation) zu finden. Dennoch ist es
ein sehr interessantes Build-System, nicht zuletzt wegen der sehr intelligenten
Änderungserkennung und der mächtigen Programmiersprache welche es bietet.

\texttt{make} bleibt weiterhin nicht wegzudenken, nicht nur weil POSIX es
standardisiert, sondern auch weil es ein Ziel von Build-System Generatoren ist,
z.B. von CMake.


	\pagebreak
	\phantomsection
	\addcontentsline{toc}{section}{Literatur}
	\noindent
	\begin{minipage}{\textwidth}
		\nocitegnumakebib{GNU_Make_manual}
\nocitegnumakebib{OreillyMake}
\nocitegnumakebib{DobbsDebuggingMakefiles}
\nocitegnumakebib{JGC_GMD}
\bibliographystylegnumakebib{plain}
\bibliographygnumakebib{../build_systems}

	\end{minipage}
	\begin{minipage}{\textwidth}
		\nociteomakebib{OMakeManual}
\bibliographystyleomakebib{plain}
\bibliographyomakebib{../build_systems.bib}

	\end{minipage}
	\begin{minipage}{\textwidth}
		\nociteautotoolsbib{CalcoteAutotools}
\nociteautotoolsbib{AutomakeManual}
\bibliographystyleautotoolsbib{plain}
\bibliographyautotoolsbib{../build_systems.bib}

	\end{minipage}
	\begin{minipage}{\textwidth}
		\nocitecmakebib{CMakeWiki}
\nocitecmakebib{CMakeDocumentation}
\nocitecmakebib{aosabook}
\bibliographystylecmakebib{plain}
\bibliographycmakebib{../build_systems.bib}

	\end{minipage}
\end{document}
