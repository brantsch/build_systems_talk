\documentclass[a4paper]{report}
\usepackage[ngerman]{babel}
\usepackage[T1]{fontenc}
\usepackage[utf8]{inputenc}
\usepackage{fix-cm}
\usepackage{hyperref}
\usepackage{multibib}

%create parts of bibliography
\newcites{gnumakebib}{Literatur zu GNU Make}
\newcites{cmakebib}{Literatur zu CMake}
\newcites{autotoolsbib}{Literatur zu den GNU Autotools}
\newcites{omakebib}{Literatur zu OMake}

%make hyperref look a little bit more beautiful
\hypersetup{colorlinks=true,linktoc=section}

\begin{document}
	\pagenumbering{gobble}
	\begin{frame}
	\title{Build Systems für C, C++, etc.}
	\author{Peter Brantsch}
	\date{29.04.2014}
	\maketitle
\end{frame}

	\clearpage
	\tableofcontents
	\pagebreak
	\pagenumbering{arabic}

	\chapter{Einführung}
	% vim: spell spelllang=de textwidth=80
\subsection{Motivation für den Einsatz eines Build-Systems} 

Als Werkzeuge der Software-Entwicklung sind Build-Systeme schon seit Jahrzehnten
nicht wegzudenken, denn sie erleichtern und verbessern die Arbeit der
Software-Entwickler erheblich, beziehungsweise erlauben den Entwicklern große
und komplexe Projekte überhaupt zu bewältigen. 
%
Durch den Einsatz eines Build-Systems sparen sich Entwickler viel Zeit und
Denk-Arbeit. Statt für die Erstellung bzw. Neuerstellung eines Projekts viele
Aufrufe diverser Programme manuell vornehmen zu müssen, sind beim Einsatz eines
Build-Systems nur noch wenige Aufrufe desselben nötig, in vielen Fällen auch
nur ein einziger.
%
Weitere Vereinfachungen, bestehend in projektübergreifend einheitlichen Befehlen
für das Build-System ergeben sich durch Konventionen für die Konfiguration, so
dass für das Erstellen vieler verschiedener Projekte einheitliche Aufrufe des
Build-Systems genügen.

Fortgeschrittene Build-Systeme unterstützen den Entwickler auch bei der
Portierung von Software-Projekten auf mehrere Plattformen. %FIXME

Außerdem beschleunigen Build-Systeme den Erstellungsprozess ungemein.
Build-Systeme führen die benötigten Aktionen in erheblich rascherer Folge aus
als ein Kommandos in eine Shell tippender Benutzer. 
%
Darüberhinaus sind selbst rudimentäre Build-Systeme bereits in der Lage unnötige
Erstellung/Neuerstellung von Teilen eines Programms zu vermeiden und sparen
dadurch Rechenzeit beim Erstellungsprozess.
%
Weiter beschleunigt werden kann die Erstellung durch automatische
Parallelisierung, bis hin zu verteilter Erstellung auf einer Build-Farm.

\subsection{Funktion eines Build-Systems}

Die offensichtlichste Funktion eines Build-Systems ist die Automatisierung der
Erstellung von Dateien. Dies schließt die Verfolgung von Abhängigkeiten ein,
d.h. dass das Build-System bei der Entscheidung ob eine Datei (neu) erstellt
werden soll berücksichtigt von welchen anderen Dateien diese abhängt.
%
Dabei gehen die Systeme rekursiv vor und erzeugen für die Erstellung einer Datei
benötigte Ressourcen, falls diese noch nicht vorliegen oder veraltet
\footnote{besitzen ihrerseits wiederum Abhängigkeiten, die neuer sind als sie
selbst} sind.

Ab einer gewissen Größe ergeben sich nennenswerte
Konfigurationsmöglichkeiten für Projekte, beispielsweise für das Ein- oder
Ausschalten von Features. Hinreichend fortgeschrittene Build-Systeme
unterstützen dies selbstverständlich. Konfigurationsvariablen werden
beispielsweise abhängig von Eigenschaften des Host-Systems gesetzt, oder auch
durch Eingaben des Benutzers, z.B. in einem Konfigurations-Dialog oder in einer
Konfigurationsdatei.

%
Zum Funktionsumfang fortgeschrittener Build-Systeme gehört unter anderem auch
die Ausführung von Tests zur Sicherstellung der erfolgreichen Erstellung. Wie
bereits erwähnt können Tests auch das Host-System untersuchen, z.B. um zu
überprüfen ob dieses über für einen erfolgreichen Build benötigte Features
verfügt. Nach abgeschlossener Erstellung kann des weiteren auch das
Build-Ergebnis getestet werden. Es bietet sich hier offensichtlich an, das
gerade erstellte Programm auf seine korrekte Funktion zu überprüfen.

Nicht notwendig, aber hilfreich, und in vielen Build-Systemen vorhanden ist
Unterstützung für Paketierung und/oder Installation des Build-Ergebnisses. So
sind einige Build-Systeme beispielsweise in der Lage, installierbare Pakete für
verschiedene Paketmanager/Installationsroutinen zu erstellen, oder
Build-Ergebnisse gleich auf dem Host-System in die richtigen Verzeichnisse zu
installieren.


	\chapter{GNU Make}

	\chapter{Fortschrittliche Build-Systeme}
		\section{OMake}
		\section{GNU Autotools}
		\section{CMake}
	%\chapter*{Literatur}
	\addcontentsline{toc}{chapter}{Literatur}
	%\section*{Literatur zu GNU Make}
		\nocitegnumakebib{GNU_Make_manual}
\nocitegnumakebib{OreillyMake}
\nocitegnumakebib{DobbsDebuggingMakefiles}
\nocitegnumakebib{JGC_GMD}
\bibliographystylegnumakebib{plain}
\bibliographygnumakebib{../build_systems}

	%\section*{Literatur zu OMake}
		\nociteomakebib{OMakeManual}
\bibliographystyleomakebib{plain}
\bibliographyomakebib{../build_systems.bib}

	%\section*{Literatur zu den GNU Autotools}
		\nociteautotoolsbib{CalcoteAutotools}
\nociteautotoolsbib{AutomakeManual}
\bibliographystyleautotoolsbib{plain}
\bibliographyautotoolsbib{../build_systems.bib}

	%\section*{Literatur zu CMake}
		\nocitecmakebib{CMakeWiki}
\nocitecmakebib{CMakeDocumentation}
\nocitecmakebib{aosabook}
\bibliographystylecmakebib{plain}
\bibliographycmakebib{../build_systems.bib}

\end{document}
