% vim: spell spelllang=de
\documentclass[a4paper,12pt]{report}
\usepackage[ngerman]{babel}
\usepackage[T1]{fontenc}
\usepackage[utf8]{inputenc}
\usepackage{fix-cm}
\usepackage{hyperref}
\usepackage{multibib}

%create parts of bibliography
\newcites{gnumakebib}{Literatur zu GNU Make}
\newcites{cmakebib}{Literatur zu CMake}
\newcites{autotoolsbib}{Literatur zu den GNU Autotools}
\newcites{omakebib}{Literatur zu OMake}

%make hyperref look a little bit more beautiful
\hypersetup{colorlinks=true,linktoc=section}

%re-style description item labels
\renewcommand*{\descriptionlabel}[1]{%
	\hspace\labelsep%
	\normalfont\underline{#1}%
}

\begin{document}
	\pagenumbering{gobble}
	\begin{frame}
	\title{Build Systems für C, C++, etc.}
	\author{Peter Brantsch}
	\date{29.04.2014}
	\maketitle
\end{frame}

	\clearpage
	\tableofcontents
	\pagebreak
	\pagenumbering{arabic}
	\addtocounter{page}{2}

	\chapter{Einführung}
	% vim: spell spelllang=de textwidth=80
\subsection{Motivation für den Einsatz eines Build-Systems} 

Build-Systeme erleichtern und verbessern die Arbeit von Software-Entwicklern
erheblich, bzw. erlauben ihnen große und komplexe Projekte überhaupt zu
bewältigen. Sie sparen viel Arbeit, da mit einem Build-System zur Erstellung des
Programms nicht mehr viele Aufrufe diverser Programme nötig sind, sondern
idealerweise nur noch ein Aufruf des Build-Systems.
%
Aufgrund von Konventionen für die Konfiguration von Build-Systemen sind zudem
die Argumente dieser Aufrufe oft projektübergreifend gleich.

Darüberhinaus erlauben Build-Systeme eine zum Teil drastische Beschleunigung des
Übersetzungprozesses, da sie die benötigten Aktionen in deutlich rascherer Folge
ausführen als dies manuell möglich wäre, und im Unterschied zu einem simplen
Build-Skript die unnötige Erstellung/Neuerstellung vermeiden können. Auch sind
die Systeme z.T. in der Lage die Erstellung automatisch zu parallelisieren, bis
hin zur verteilten Erstellung auf einer Build-Farm.

\subsection{Funktion eines Build-Systems}

Die offensichtlichste Funktion eines Build-Systems ist die Automatisierung der
Erstellung von Dateien. Dies schließt die Verfolgung von Abhängigkeiten ein,
d.h. dass das Build-System bei der Entscheidung ob eine Datei (neu) erstellt
werden soll berücksichtigt von welchen anderen Dateien diese abhängt.
%
Dabei gehen die Systeme rekursiv vor und erzeugen für die Erstellung einer Datei
benötigte Ressourcen, falls diese noch nicht vorliegen oder veraltet
\footnote{besitzen ihrerseits wiederum Abhängigkeiten, die neuer sind als sie
selbst} sind.

Ab einer gewissen Größe ergeben sich nennenswerte Konfigurationsmöglichkeiten
für Projekte, beispielsweise für das Ein- oder Ausschalten von Features.
Konfigurationsvariablen werden beispielsweise abhängig von Eigenschaften des
Host-Systems gesetzt, oder auch durch Eingaben des Benutzers, z.B. in einem
Dialog oder in einer Konfigurationsdatei.

%
Zum Funktionsumfang fortgeschrittener Build-Systeme gehört unter anderem auch
die Ausführung von Tests zur Sicherstellung der erfolgreichen Erstellung. Wie
bereits erwähnt können Tests auch das Host-System untersuchen, z.B. um zu
überprüfen ob dieses über für einen erfolgreichen Build benötigte Features
verfügt. Nach abgeschlossener Erstellung kann des Weiteren auch das
Build-Ergebnis getestet werden. Es bietet sich hier offensichtlich an, das
gerade erstellte Programm auf seine korrekte Funktion zu überprüfen.

Nicht notwendig, aber hilfreich, und in vielen Build-Systemen vorhanden ist
Unterstützung für Paketierung und/oder Installation des Build-Ergebnisses. So
sind einige Build-Systeme beispielsweise in der Lage, installierbare Pakete für
verschiedene Paketmanager/Installationsroutinen zu erstellen, oder
Build-Ergebnisse gleich auf dem Host-System in die richtigen Verzeichnisse zu
installieren.


	\chapter{GNU Make}

	\chapter{Fortschrittliche Build-Systeme}
		\section{OMake}
		\section{GNU Autotools}
		\section{CMake}

	\pagebreak
	\pagenumbering{Roman}
	\addcontentsline{toc}{chapter}{Literatur}
	\begin{minipage}{\textwidth}
		\nocitegnumakebib{GNU_Make_manual}
\nocitegnumakebib{OreillyMake}
\nocitegnumakebib{DobbsDebuggingMakefiles}
\nocitegnumakebib{JGC_GMD}
\bibliographystylegnumakebib{plain}
\bibliographygnumakebib{../build_systems}

	\end{minipage}
	\begin{minipage}{\textwidth}
		\nociteomakebib{OMakeManual}
\bibliographystyleomakebib{plain}
\bibliographyomakebib{../build_systems.bib}

	\end{minipage}
	\begin{minipage}{\textwidth}
		\nociteautotoolsbib{CalcoteAutotools}
\nociteautotoolsbib{AutomakeManual}
\bibliographystyleautotoolsbib{plain}
\bibliographyautotoolsbib{../build_systems.bib}

	\end{minipage}
	\begin{minipage}{\textwidth}
		\nocitecmakebib{CMakeWiki}
\nocitecmakebib{CMakeDocumentation}
\nocitecmakebib{aosabook}
\bibliographystylecmakebib{plain}
\bibliographycmakebib{../build_systems.bib}

	\end{minipage}
\end{document}
