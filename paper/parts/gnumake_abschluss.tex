% vim: spell spelllang=de
\section{Abschluss}

Nichts auf dieser Welt hat nur Vor- oder Nachteile. Im Folgenden soll nun kurz
auf die Vor- und Nachteile von GNU Make eingegangen werden.

\subsection{Stärken}

% Allgegenwärtigkeit auf POSIX-Systemen
Für \texttt{make} spricht zunächst ein mal dessen weite Verbreitung. Möchte man
sich der Verfügbarkeit eines Build-Systems für ein beliebiges UNIX(-ähnliches)
System sicher sein, so kann man getrost zu \texttt{make} greifen, denn dieses
ist Teil des POSIX-Standards\citegnumakebib{OpenGroupBaseSpec}, was jedoch -- in einem kleinen Vorgriff auf den
Abschnitt über Nachteile von \texttt{make} -- nicht den Eindruck vermitteln
sollte \texttt{Makefiles} wären immer portabel.

% Einfachheit des Einstiegs für viele Anwendungsfälle ausreichende Mächtigkeit
Die verglichen mit einigen anderen Build-Systemen rudimentäre Natur von
\texttt{make} macht des Weiteren den Einstieg leicht. Das \texttt{Makefile} für
ein kleines Projekt ist schnell geschrieben oder noch schneller kopiert.

% Flexibilität
Für viele Standard-Aufgaben ist \texttt{make} außerdem flexibel genug. Die
integrierten Funktionen sowie die Möglichkeit beliebige Programme aufzurufen
erlauben den Einsatz von \texttt{make} für diverseste Aufgaben bei denen Dateien
abhängig von anderen Dateien aktualisiert werden sollen.

\subsection{Schwächen} 

Das Build-System, dessen Namen ich inzwischen schon viel zu oft ausgeschrieben
habe, ist bei weitem nicht der Weisheit letzter Schluss. Deswegen werden wir
uns auch in Kapitel~\ref{chapter:advancedbs} mit Werkzeugen befassen, die
deutlich fortgeschrittener sind als \texttt{make}. Nicht ohne Grund ist auch die
Liste der Schwächen deutlich länger als die der Stärken.

%Debugging
Läuft ein Erstellungsprozess mit \texttt{make} nicht wie erwartet ab, ist man in
der Verlegenheit ein \texttt{Makefile} debuggen zu müssen. Wie dies geschehen
kann ist zwar recht gut dokumentiert \citegnumakebib{DobbsDebuggingMakefiles},
und es existiert sogar eine Art interaktiver Debugger \citegnumakebib{JGC_GMD},
dennoch ist dies bei weitem nicht so bequem wie das Debugging richtiger
Programmiersprachen.

Gelegentlich könnte man in die Verlegenheit kommen, in einem \texttt{Makefile}
Programme aufzurufen, welche mehrere Dateien gleichzeitig
verändern.\footnote{\LaTeX~z.B. erzeugt einen Wust von Dateien, von
denen einige von externen Tools gelesen werden müssen um ein Dokument korrekt zu
erzeugen.}%
Zu Ärger und/oder Enttäuschung führt in solchen Fällen \texttt{make}s fehlende
Unterstützung für Regeln die mehrere Targets gleichzeitig aktualisieren.

%Änderungserkennung
Manchmal wird sich auch die Änderungserkennung, welche sich auf das
Änderungsdatum der Dateien verlässt, als unpraktisch erweisen, denn oft genug
bedeutet ein neuer Wert des Änderungsdatums nicht auch eine Änderung des
Dateiinhalts. 
%Gravierender wirkt sich diese Schwäche im Zusammenhang mit
%Netzwerkdateisystemen und einer falsch gehenden Systemuhr aus.

%Geschwätzigkeit
Aus der rudimentären Natur von \texttt{make}, die seine Grundzüge so einfach und
leicht erlernbar macht, erwächst auch ein Nachteil, nämlich eine gewisse
„Geschwätzigkeit“ von \texttt{Makefile}s für Standardaufgaben. Da gewisse
Targets (z.B. \texttt{install}, \texttt{clean})
\citegnumakebib{GNUCodingStandards} oft benötigt werden, aber nicht eingebaut
sind, müssen sie jedes mal inklusive Regel und Rezept vom Entwickler geschrieben
werden.

%keine Out-Of-Source Builds
Weiterhin sind keine \emph{Out-Of-Source Builds} wie z.B. die gleichzeitige
Erstellung eines normalen und eines Debug-Builds in separaten Verzeichnissen
außerhalb des Quellcodeverzeichnisses möglich, d.h. die Erstellung
von Programmen muss, falls man nicht gewillt ist unschöne Workarounds zu
programmieren, im Quellcode-Verzeichnisbaum erfolgen.

%Mangel an Portabilität
Darüberhinaus besteht ein Mangel an Portabilität. Zwar ist \texttt{make}
standardisiert, aber Implementierungen wie z.B. GNU Make definieren umfangreiche
Erweiterungen.  
%
Selbst wenn es gelingt ein \texttt{Makefile} zu schreiben, das
in allen Implementierungen von \texttt{make} gültig ist, welche auf den durch
ein Projekt unterstützten Plattformen laufen, können diese Systeme immer noch
genügend Eigenheiten haben, wie beispielsweise verschiedene Arten mit
dynamischen Bibliotheken umzugehen.
