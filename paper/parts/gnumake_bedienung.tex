% vim: spell spelllang=de 
% vim: set textwidth=80
\section{Bedienung und Konfiguration} 

Bevor ein Projekt mit \texttt{make} automatisch erstellt werden kann, muss erst
die Konfiguration für selbiges in Form eines \texttt{Makefile} angelegt werden,
welches \texttt{make} darüber informiert, welche Dateien es wie aktualisieren
soll, abhängig von welchen anderen Dateien. GNU Make (und übrigens auch andere
Implementierungen von Make) werden in dem Verzeichnis in dem sie aufgerufen
werden nach einer Datei dieses Namens suchen und sie einlesen.
%
Ist diese Konfiguration erst ein mal vorhanden erfordert die Erstellung des
Projekts vom Benutzer nur noch mindestens einen Aufruf von \texttt{make}.

\subsection{Makefiles}

\subsubsection{Bestandteile} 

\begin{description}
%
	\item[Regeln] Beschreiben wann und wie Dateien (\emph{Targets} der Regel)
	abhängig von \emph{Prerequisites} erstellt werden sollen.
	%
	\begin{description}
		\item[explizit] Targets benennen Dateien
		\item[implizit] Targets benennen Klassen von Dateien
	\end{description}
	%
%
	\item[Direktiven] unter anderem:
	%
	\begin{itemize}
		\item Einlesen anderer Makefiles
		\item Kontrollstrukturen für Entscheidungen
		\item Definition mehrzeiliger Variablen
	\end{itemize}
%
	\item[Variablendefinitionen] Zuweisungen von Zeichenketten zu Namen
%
	\item[Kommentare] von \texttt{make} zu ignorierender Text
\end{description}
