% vim: spell spelllang=de
% vim: textwidth=80
\section{Makefiles}

Bevor ein Projekt mit \texttt{make} automatisch erstellt werden kann, muss erst
die Konfiguration für selbiges in Form eines \texttt{Makefile} angelegt werden.
GNU Make (und übrigens auch andere Implementierungen von Make) werden in dem
Verzeichnis in dem sie aufgerufen werden nach einer Datei dieses Namens suchen
und sie einlesen.

\subsection{Bestandteile}

\begin{description}
%
	\item[Regeln] Beschreiben wann und wie Dateien (\emph{Targets} der Regel)
	abhängig von \emph{Prerequisites} erstellt werden sollen.
	%
	\begin{description}
		\item[explizit] Targets benennen Dateien
		\item[implizit] Targets benennen Klassen von Dateien
	\end{description}
	%
%
	\item[Direktiven] unter anderem:
	%
	\begin{itemize}
		\item Einlesen anderer Makefiles
		\item Kontrollstrukturen für Entscheidungen
		\item Definition mehrzeiliger Variablen
	\end{itemize}
%
	\item[Variablendefinitionen] Zuweisungen von Zeichenketten zu Namen
%
	\item[Kommentare] von \texttt{make} zu ignorierender Text
\end{description}

\subsection{Syntax}

	\subsubsection*{Kommentare}  
		%
		\begin{syntdiag}
	‘\#’
	\synt{text}
	\synt{newline}
\end{syntdiag}

		%
		Sobald \texttt{make} ein `\#'-Zeichen liest, ignoriert es den Rest der Zeile
		bis einschließlich des Zeilenumbruchs.

	\subsubsection*{Strings}  
		%
		\begin{syntdiag}
	\begin{rep}
		\begin{stack}
			\tok{ASCII Zeichen ohne \$} \\
			\$\$ \\
			\synt{variablenreferenz} \\
			\synt{funktionsaufruf}
		\end{stack}
	\end{rep}
\end{syntdiag}

		%
		GNU Make kennt keine anderen Datentypen als Strings.

	\subsubsection*{Funktionsaufrufe}  \begin{syntdiag}
	‘\$(’ \synt{name}
	\synt{whitespace}
	\begin{stack}
		\\
		\begin{rep}
			\synt{string} ‘,’
		\end{rep}
	\end{stack}
	\synt{string}
	‘)’
\end{syntdiag}

		%
		GNU Make stellt eine Sammlung von Hilfsfunktionen zur Verfügung, welche
		diversen Zwecken dienen, wie beispielsweise der Manipulation von
		Strings, der Umsetzung weiterer Bedingungs-Konstrukte, wiederholten
		Auswertung von Ausdrücken, etc.  \footnote{Siehe auch: Kapitel 8 (Functions
		for Transforming Text) in der offiziellen Dokumentation von GNU Make
		\citegnumakebib{GNU_Make_Manual}}

		Sowohl Funktionsaufrufe als auch Variablenreferenzen sind gleichermaßen
		mit runden Klammern wie mit geschweiften Klammern möglich, allerdings
		empfiehlt der Autor sich für jeweils eine Art von Klammern zu
		entscheiden und dabei zu bleiben.

	\subsubsection*{Entscheidungs-Kontrollstrukturen}	
		%
		\begin{syntdiag}
	\synt{bedingung}
	\synt{string}
	\begin{stack}
		\\
		\begin{stack}
			‘else’ \synt{string} \\
			\begin{rep}
				\synt{bedingung}
				\synt{string}
			\end{rep}
		\end{stack}
	\end{stack}
	‘endif’
\end{syntdiag}


	\subsubsection*{Bedingungen}  
		%
		\begin{syntdiag}
	\begin{stack}
		‘ifeq’ \\
		‘ifneq’ 
	\end{stack}
	\begin{stack}
		‘(’ \synt{string} ‘,’ \synt{string} ‘)’ \\
		\begin{stack}
			‘''’ \synt{string} ‘''’ \\
			‘'’ \synt{string} ‘'‘
		\end{stack}
		\begin{stack}
			‘''’ \synt{string} ‘''‘ \\
			‘'’ \synt{string} ‘'’
		\end{stack}
	\end{stack}
\end{syntdiag}

\begin{syntdiag}
	\begin{stack}
		‘ifdef’ \\
		‘ifndef’
	\end{stack}
	\synt{variable~name}
\end{syntdiag}


	\subsubsection*{Variablenreferenzen}  
		%
		\begin{syntdiag}
	\$ \begin{stack}
		\{ \synt{variable~name} \} \\
		( \synt{variable~name} )
	\end{stack}
\end{syntdiag}

		%
		\fcolorbox{red}{white}{
			\parbox{\linewidth-4\fboxsep}{
				Achtung: Referenzen auf undefinierte Variablen
				werden \emph{ohne Fehlermeldung} zu leeren Strings aufgelöst!
			}
		}
		\noindent
		Dieses Verhaltens sollte man sich bei der Fehlersuche in einem
		komplizierteren \texttt{Makefile} bewusst sein.

	\subsubsection*{Definition/Redefinition einer Variable}
		%
		\begin{syntdiag}
	\synt{variable~name}
	\begin{stack}
		‘=’ \\
		‘:=’
	\end{stack}
	\begin{rep}
		\synt{string}
	\end{rep}
	\synt{newline}
\end{syntdiag}
\begin{syntdiag}
	‘define’
	\synt{variable~name}
	\begin{stack}
		‘=’\\
		‘:=’
	\end{stack}
	\synt{newline}
	\begin{rep}
		\synt{string}
		\synt{newline}
	\end{rep}
	‘endef’
	\synt{newline}
\end{syntdiag}

		%
		\texttt{make} unterscheidet zwei Arten von Variablen, denen Werte
		explizit zugewiesen werden können: \emph{rekursiv expandierte Variablen}
		und \emph{einfach expandierte Variablen}.
		%
		Bei der Definition einer rekursiv expandierten Variable geschieht noch
		keinerlei Expansion von Variablenreferenzen. Diese erfolgt erst, wenn
		die Variable referenziert wird. Dann werden, wie der Name schon
		andeutet, Referenzen rekursiv expandiert.
		%
		Referenziert man hingegen in der Definition einer einfach expandierten
		Variable andere Variablen, so erfolgt die Expansion dieser Referenzen
		sofort. Gelegentlich ist das hilfreich, beispielsweise wenn man in der
		Definition einer Variable diese selbst referenzieren
		will\footnote{verwendete man eine rekursiv expandierte Variable
		entstünde hier eine Endlosschleife}, meistens werden aber normale,
		rekursiv expandierte Variablen genügen.
		
		Des Weiteren existieren auch sogenannte \emph{automatische Variablen},
		welche nicht explizit zugewiesen werden können, sondern ihre Werte von
		\texttt{make} erhalten.

	\subsubsection*{Anhängen an eine Variable}
		%
		\begin{syntdiag}
	\synt{variable~name}
	‘+=’
	\begin{rep}
		\synt{string}
	\end{rep}
\end{syntdiag}

		%	
		Der aufmerksame Leser wird beim Lesen des Beispiel-\texttt{Makefile}s
		auf Seite~\pageref{subsubsection:examplemakefile} feststellen, dass
		auch das Anhängen an undefinierte Variablen möglich ist. Es hat den
		selben Effekt wie eine normale Zuweisung.

	\subsubsection*{Regeln}
		%
		\begin{syntdiag}
	\begin{rep}
		\synt{target}
	\end{rep}
	\begin{stack}
		‘:’ \\
		‘::’
	\end{stack}
	\begin{stack}
		\begin{rep}
			\synt{prerequisite}
		\end{rep} \\
	\end{stack}
	\begin{stack}
		\\
		\synt{newline}
		\begin{rep}
			\synt{recipe line}
		\end{rep}
	\end{stack}
\end{syntdiag}

		%
		\textit{Recipe Lines} \begin{syntdiag}
	\begin{stack}
		\\
		\begin{rep}
			\synt{tab}
			\synt{string}
			‘\textbackslash’
			\synt{newline}
		\end{rep}
	\end{stack}
	\synt{tab}
	\synt{string}
	\synt{newline}
\end{syntdiag}

		%
		\textit{Targets} \begin{syntdiag}
	\begin{stack}
		\synt{dateiname} \\
		\synt{pattern}
	\end{stack}
\end{syntdiag}

		%
		\textit{Prerequisites} \begin{syntdiag}
	\begin{stack}
		\synt{anderes~target} \\
		\synt{dateiname} \\
		\synt{pattern}
	\end{stack}
\end{syntdiag}

		%
		\textit{Patterns} \begin{syntdiag}
	\begin{stack}
		\\
		\synt{string~ohne~whitespace}
	\end{stack}
	‘\%’
	\begin{stack}
		\\
		\synt{string~ohne~whitespace}
	\end{stack}
\end{syntdiag}


		Regeln bestimmen wie Dateien, die sogenannten \emph{Targets} einer
		Regel, zu aktualisieren sind, und von welchen anderen Dateien sie
		abhängen.
		
		Ist eine der Abhängigkeiten der Datei (im Syntaxdiagramm: Prerequisites)
		neuer als die Datei, so betrachtet \texttt{make} die Datei als veraltet
		und wird das Rezept ausführen um die Datei neu zu erstellen. Hierzu wird
		es jede Zeile (oben \textit{recipe line} genannt) von der Shell
		ausführen lassen.

		Recipe Lines sind in diesem Kontext als logische Zeilen zu verstehen.
		Eine logische Zeile kann sich über mehrere tatsächliche Zeilen
		erstrecken, falls alle Zeilenumbrüche, mit Ausnahme dessen der die
		logische Zeile beendet, mit Backslashes escaped werden.

		\clearpage
		\subsection{Ein Beispiel-\texttt{Makefile}}
			\label{subsubsection:examplemakefile}

			Hierbei handelt es sich um das \texttt{Makefile}, welches uns
			bereits im GdP1-Praktikum im ersten Semester begegnet ist.
			\footnote{Erstellt von Prof. Dr. Franz Regensburger} Damals diente
			es zur Übersetzung der Testate ab einschließlich \texttt{Worm080}.
			Mit der vorliegenden Beschreibung der Syntax und Funktionsweise von
			\texttt{make} sollte es leicht zu verstehen sein.

			{
			\footnotesize
			\inputminted[linenos=true,stepnumber=5]{make}{../code/Worm080_Makefile}
			}

\section{Bedienung}

	Die Grundlagen der Bedienung könnten einfacher kaum sein.  Zur Erstellung
	eines Programms genügt im einfachsten Fall ein Aufruf von \texttt{make} ohne
	Kommandozeilenargumente im Projektordner.
	%
	Die Namen eines oder mehrerer Targets können als Kommandozeilenargumente
	angegeben werden, mit der Folge dass diese anstelle des Standard-Targets
	bearbeitet werden.
