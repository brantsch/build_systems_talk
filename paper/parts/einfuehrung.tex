\section{Motivation für den Einsatz eines Build-Systems} 

Als Werkzeuge der Software-Entwicklung sind Build-Systeme schon seit Jahrzehnten
nicht wegzudenken, denn sie erleichtern und verbessern die Arbeit der
Software-Entwickler erheblich, beziehungsweise erlauben den Entwicklern große
und komplexe Projekte überhaupt zu bewältigen. 
%
Durch den Einsatz eines Build-Systems sparen sich Entwickler viel Zeit und
Denk-Arbeit. Statt für die Erstellung bzw. Neuerstellung eines Projekts viele
Aufrufe diverser Programme manuell vornehmen zu müssen, sind beim Einsatz eines
Build-Systems nur noch wenige Aufrufe des desselben nötig, in vielen Fällen auch
nur ein einziger.
%
Weitere Vereinfachungen, bestehend in projektübergreifend einheitlichen Befehlen
für das Build-System ergeben sich durch Konventionen für die Konfiguration, so
dass für das Erstellen vieler verschiedener Projekte einheitliche Aufrufe des
Build-Systems genügen.

Fortgeschrittene Build-Systeme unterstützen den Entwickler auch bei der
Portierung von Software-Projekten auf mehrere Plattformen. %FIXME

Außerdem beschleunigen Build-Systeme den Erstellungsprozess ungemein.
Build-Systeme führen die benötigten Aktionen in erheblich rascherer Folge aus
als ein Kommandos in eine Shell tippender Benutzer. 
%
Darüberhinaus sind selbst rudimentäre Build-Systeme bereits in der Lage unnötige
Erstellung/Neuerstellung von Teilen eines Programms zu vermeiden und sparen
dadurch Rechenzeit beim Erstellungsprozess.
%
Weiter beschleunigt werden kann die Erstellung durch automatische
Parallelisierung, bis hin zu verteilter Erstellung auf einer Build-Farm.

\section{Funktion eines Build-Systems}
