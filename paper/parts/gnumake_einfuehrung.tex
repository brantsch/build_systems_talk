% vim: spell spelllang=de textwidth=80
%\section{Einführung}
Das Programm \texttt{make} entstand im Umfeld von UNIX und ist Teil des POSIX
Standards.  Seine Aufgabe ist zu erkennen welche Teile eines großen Programms
sich geändert haben, und die Kommandos abzusetzen um diese zu aktualisieren.
\citegnumakebib{GNU_Make_Manual}
%
Im Folgenden soll von GNU Make die Rede sein, bei dem es sich aufgrund der
Popularität von GNU/Linux heutzutage wohl um das verbreitetste \texttt{make}
handeln dürfte.
