% vim: spell spelllang=de textwidth=80
\newenvironment{bstable}{%
	\definecolor{lightgray}{gray}{0.9}%
	\rowcolors{1}{white}{lightgray}%
	\begin{tabular}{%
		p{0.2\textwidth}%
		|%
		p{0.16\textwidth}%
		p{0.16\textwidth}%
		p{0.16\textwidth}%
		p{0.16\textwidth}%
	}%
	& \textbf{(GNU) Make} & \textbf{OMake} & \textbf{CMake} & \textbf{Autotools} \\ \hline%
}{%
	\end{tabular}%
}

Die folgende Übersicht ist bei weitem nicht erschöpfend. Die für den Vergleich
herangezogenen Features wurden vom Autor für wichtig und/oder praktisch
erachtet.

\noindent
\begin{bstable}
	Plattfor\-men 				 & POSIX & von OCaml unterstützte & POSIX, Windows, OS/2, etc. & POSIX \\
Unterstützung bei Cross-Compilation & \tblxmark & \tblxmark & \tblcmark & \tblcmark \\
Unterstützung bei Portierung & \tblxmark & \tblcmark & \tblcmark & \tblcmark \\

	interaktive Konfigu\-ration des Projekts 	& \tblxmark & \tblxmark & \tblcmark & \tblcmark \\
Autokon\-figuration 						& \tblxmark & \tblcmark & \tblcmark & \tblcmark \\
Test\-suite Unterstützung 					& \tblxmark & \tblxmark & \tblcmark & \tblcmark \\

	Erkennung von Änderungen 	& Zeit\-stempel & MD5 Digest & \tblcenter{???} & Zeitstempel\\
Abhängigkeits\-analyse		& \tblxmark & \tblcmark & \tblcmark & \tblcmark \\

	Parallele Erstellungs\-schritte & \tblcmark	& \tblcmark	& \tblcmark & \tblcmark \\
Fort\-schritts\-anzeige 	& \tblxmark & \tblcmark & \tblcmark & \tblxmark \\
Out-Of-Source Builds 		& \tblxmark & \tblcmark & \tblcmark & \tblcmark \\

\end{bstable}

\begin{itemize}
%
\item Unter \emph{Cross-Compilation} versteht man die Übersetzung von Software
auf einem System dessen Architektur von der des Systems auf welchem die Software
ausgeführt werden soll verschieden ist.
%
\item Im Zusammenhang dieses Vergleichs steht \emph{Unterstützung bei der
Portierung} für die Abstraktion von Eigenheiten der Zielsysteme durch das
Build-System, z.B. des Umgangs mit Bibliotheken.
%
\item Bei der \emph{interaktiven Konfiguration des Projekts} erfragt das
Build-System die gewünschten Werte für Konfigurationsvariablen über ein
Dialog-Interface von seinem Anwender.
%
\item \emph{Autokonfiguration} ist die automatische Erkennung von und Anpassung
an Gegebenheiten auf dem Zielsystem.
%
\item Build-Systeme, welche \emph{Testsuites} unterstützen, erlauben die
Ausführung von Tests des Build-Ergebnisses.
%
\item \emph{Änderungserkennung} beschreibt, anhand welcher Kriterien ein
Build-System bestimmt ob eine Datei geändert wurde.

(Anmerkung: Für CMake enthält die Tabelle keine Angabe, da es bei CMake, welches
ein Build-System-Generator ist, stark davon abhängt welches native Build-System
verwendet wird.)
%
\item Eine Alternative zum mühsamen und fehleranfälligen Eintragen von
Abhängigkeiten per Hand ist die \emph{Abhängigkeitsanalyse}, bei welcher das
Build-System auswertet von welchen anderen Dateien eine Datei abhängt, z.B.
welche Header-Dateien eine C-Quellcodedatei inkludiert.
%
\item Wichtig besonders für größere Projekte sind \emph{parallele
Erstellungsschritte}. Ist dem Build-System per Kommandozeilenargument oder
Konfigurationsoption erlaubt, die Übersetzung zu parallelisieren, so wird es
versuchen voneinander unabhängige Erstellungsschritte parallel auszuführen.
Insbesondere auf Mehrprozessorsystemen, welche heutzutage sehr verbreitet sind,
lässt sich hierdurch die Erstellung z.T. deutlich beschleunigen. Die logische
Fortsetzung ist die verteilte Erstellung auf einer Build-Farm.
%
\item \emph{Out-Of-Source Builds}\footnote{gelegentlich auch \emph{Out-Of-Place
Builds} genannt} sind Erstellungen, welche außerhalb des Quellcodeverzeichnisses
eines Projekts erfolgen. Bei diesen liegen die Build-Konfiguration und das
Build-Ergebnis in einem vom Quellcodeverzeichnis separaten Verzeichnis. Sie sind
beispielsweise hilfreich wenn man gleichzeitig einen Build mit Debug-Symbolen
haben möchte und einen ohne.
%
\end{itemize}

\clearpage
\subsubsection*{Fazit}
%
Da CMake deutlich weniger kompliziert zu bedienen ist als die Autotools,
allerdings einen ähnlichen Funktionsumfang bietet, sollte man nach Ansicht des
Autors Abstand von den Autotools halten. Neben diesen wirkt CMake deutlich
fortschrittlicher, und in \textit{The Architecture of Open Source Applications}
\citecmakebib{aosabook} wird auch CMake als Ersatz für den alternden Ansatz mit
autoconf/libtool bezeichnet.

OMake sollte mit Vorsicht genossen werden, denn zumindest dem Autor fiel es
schwer Literatur (außer der offiziellen Dokumentation) zu finden. Dennoch ist es
ein sehr interessantes Build-System, nicht zuletzt wegen der sehr intelligenten
Änderungserkennung und der mächtigen Programmiersprache welche es bietet.

\texttt{make} bleibt weiterhin nicht wegzudenken, nicht nur weil POSIX es
standardisiert, sondern auch weil es ein Ziel von Build-System Generatoren ist,
z.B. von CMake.
